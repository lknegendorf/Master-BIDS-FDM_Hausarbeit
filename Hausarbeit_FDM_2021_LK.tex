\documentclass[12pt,a4paper,toc=bibliographynumbered,toc=indenttextentries]{scrreprt}
\usepackage[utf8]{inputenc}
\usepackage[T1]{fontenc}
\usepackage{lmodern}
\usepackage{microtype}
\usepackage{setspace}
\usepackage[ngerman]{babel}
\usepackage{doi}
\usepackage{url}
\usepackage{ccicons}
\bibliographystyle{biolett}

\author{Leonard Knegendorf}
\title{FAIRE Publikation eines Beispieldatensatzes}

\begin{document}
	\begin{titlepage}
		\centering
		\textsc{\large{Hochschule Mannheim\\Master of Science\\Biomedizinische Informatik und Data Science\\Forschungsdatenmanagement (FDM)\\Prof. Dr. Dagmar Waltemath}}\par

		\vspace{3cm}
		\textbf{\huge{FAIRE Publikation eines Beispieldatensatzes\\}}\par
	
		\vspace{3cm}
		\textsc{\large{Hausarbeit}}\par
		
		
		\vfill
		vorgelegt von\par
		\textbf{Leonard Knegendorf}\par
		Hildesheim 2021\par
	\end{titlepage}

	
	\thispagestyle{empty}
	\mbox{}
	\vfill
	\begin{center}
	\begin{tabular}{|p{.9\textwidth}|}
	\ccbysa \bigskip\par\noindent
	\small{FAIRE Publikation eines Datensatzes von Leonard Knegendorf ist lizenziert unter einer \href{http://creativecommons.org/licenses/by-sa/4.0/}{Creative Commons Namensnennung - Weitergabe unter gleichen Bedingungen 4.0 International Lizenz}.\bigskip\bigskip\par\noindent
	Beruht auf dem Datensatz:\par
	\noindent Walonoski~J, Klaus~S, Granger~E, Hall~D, Gregorowicz~A, Neyapally~G, Watson~A, Eastman~J. 2020 Sythea\texttrademark Novel coronavirus (COVID-19) model an synthetic data set.
	\newblock \emph{Intelligence-Based Medicine}~\textbf{1}, 1:100007. \newblock (\doi{10.1016/j.ibmed.2020.100007}).\par\medskip
	Download unter:
	\url{https://storage.googleapis.com/synthea-public/10k_synthea_covid19_csv.zip}.}
	\end{tabular}
	\end{center}
	\thispagestyle{empty}
	\clearpage
	
	\thispagestyle{empty}
	\tableofcontents
	\clearpage
	
	\chapter{Einleitung}
	(Text zur Beschreibung des Beispielszenarios und des Datensatzes, muss noch eingefügt werden.)\par
	Als erster Schritt, vor Beginn der Erstellung der eigentlichen Hausarbeit und der Bearbeitung des Datensatzes, erfolgte die Festlegung der Lizensierung. Die Wahl der Lizenz für die vorliegende Hausarbeit erfolgte mithilfe des \href{https://creativecommons.org/choose/}{Creative Commons License Chooser}. Anschließend wurde die System-Umgebung zur Dokumentation eingerichtet welche im Folgenden näher erläutert wird.
				
		\section{Beschreibung der System-Umgebung}
		Zu Beginn des Projektes wurde Wert auf die Einrichtung einer definierten System-Umgebung gelegt, in der die Bearbeitung des Datensatzes erfolgen kann.
		Die Versionskontrolle des bearbeiteten Datensatzes selbst erfolgt gemäß der Aufgabenstellung in einer FAIRDOMHub Sandbox. Bei FAIRDOMHub handelt es sich um ein Repositorium und eine Co-Working-Plattform die für Forschungsprojekte im Bereich der Systembiologie entwickelt wurde~\cite{10.1093/nar/gkw1032}.\par  
		Die Texdokumentation der Bearbeitungsschritte erfolgt in einem \LaTeX{}-Dokument. Hierzu wurde zunächst ein GitHub-Repositorium erstellt, mithilfe dessen eine automatisierte Versionskontrolle der Projektdokumentation erfolgen kann (\url{https://github.com/lknegendorf/Master-BIDS-FDM_Hausarbeit}). Die Gliederung des Projektdokumentation gemäß der Aufgabenstellung wurde als erste Version der Textdokumentation in das Repositorium geladen. Darüber hinaus wurde eine tabellarische Übersicht der Dateistruktur und der Versionierung des GitHub-Repositoriums in dessen Wiki erstellt (\url{https://github.com/lknegendorf/Master-BIDS-FDM_Hausarbeit/wiki/%C3%9Cbersicht-%C3%BCber-Dateistruktur-und-Versionierung}).  
		
	\chapter{Bearbeitung der Aufgabenstellung}
		
		\section*{1. FAIR-Kriterien}
	
		\section*{2. Datenqualität}
	
		\section*{3. Meta-Daten}
	
		\section*{4. Datenaufbereitung}
	
		\section*{5. Datenpublikation}
	
	\clearpage
	\singlespacing
	\bibliography{FDM_Hausarbeit-Literatur}
	\clearpage
				
	\chapter{Anhang}
	
\end{document}
